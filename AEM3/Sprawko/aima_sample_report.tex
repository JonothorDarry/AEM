\documentclass[12pt]{article}
\usepackage[utf8]{inputenc}
\usepackage[T1]{fontenc}
\usepackage{pdflscape}
\usepackage{lmodern}
\usepackage[a4paper,bindingoffset=0.2in,%
left=0.5in,right=0.5in,top=0.5in,bottom=1in,%
footskip=.25in]{geometry}
\usepackage[colorlinks=true, linkcolor=Black, urlcolor=Blue]{hyperref}
\usepackage{graphicx}
\usepackage{subcaption}
\usepackage{listings}
\usepackage{color}
\usepackage{amsmath}
\usepackage{algorithm}
\usepackage[noend]{algpseudocode}

\makeatletter
\def\BState{\State\hskip-\ALG@thistlm}
\makeatother
\usepackage{listings}
\usepackage{color}
\usepackage[table]{xcolor}
\definecolor{lightgray}{gray}{0.9}

\definecolor{codegreen}{rgb}{0,0.6,0}
\definecolor{codegray}{rgb}{0.5,0.5,0.5}
\definecolor{codepurple}{rgb}{0.58,0,0.82}
\definecolor{backcolour}{rgb}{0.95,0.95,0.92}

\lstdefinestyle{mystyle}{
	backgroundcolor=\color{backcolour},   
	commentstyle=\color{codegreen},
	keywordstyle=\color{magenta},
	numberstyle=\tiny\color{codegray},
	stringstyle=\color{codepurple},
	basicstyle=\ttfamily\footnotesize,
	breakatwhitespace=false,         
	breaklines=true,                 
	captionpos=b,                    
	keepspaces=true,                 
	numbers=left,                    
	numbersep=5pt,                  
	showspaces=false,                
	showstringspaces=false,
	showtabs=false,                  
	tabsize=2
}
\usepackage{svg}


\begin{document}
	\title{Sprawozdanie z zadania 1.\\
		\large Metody Inteligencji Sztucznej i Obliczeniowej\\
		\large Sebastian Michoń 136770}
	\date{\vspace{-10ex}}
	\maketitle
	\section{Problem}
	\begin{enumerate}
		\item Danych jest \(n\) punktów na płaszczyźnie.
		\item Punkty te będą traktowane jako wierzchołki w grafie, pomiędzy każdą parą wierzchołków istnieje krawędź, waga krawędzi między wierzchołkami \((x;y)\) jest zdefiniowana jako odległość pomiędzy tymi punktami w przestrzeni euklidesowej.
		\item Celem zadania jest znalezienie ścieżki zamkniętej przechodzącej przez \(\lceil \frac{n}{2} \rceil\) punktów minimalizującej sumę odległości pomiędzy kolejnymi parami wierzchołków w tej ścieżce.
		\item Problem należy rozwiązać heurystykami zachłannymi i heurystykami z żalem.
	\end{enumerate}
	
	\section{Pseudokod}
	\begin{algorithm}
		\caption{Trzy heurystyki - głowna funkcja}\label{erste}
		\begin{algorithmic}[1]
			\Procedure{General\_heuristic}{start\_vertex, matrix, Heuristic} 
			\State{}
			\Comment{Heuristic - wykorzystywana heurystyka}
			\State $\textit{no\_vertexes} \gets \textit{matrix.size[0]}$
			\State $\textit{pathway\_final\_len} \gets \textit{\(\lceil \frac{no\_vertexes}{2} \rceil \)}$
			\State $\textit{pathway} \gets \textit{[start\_vertex]}$
			\State $\textit{checked} \gets \textit{ARRAY[0..no\_vertices]}$
			
			\State $\textit{i} \gets \textit{0}$
			\While {\(i < no\_vertices\)} \Comment{indeksowanie od 0}
			\State $\textit{checked[i]} \gets \textit{false}$
			\State $\textit{i} \gets \textit{i+1}$
			\EndWhile
			
			\State $\textit{i} \gets \textit{1}$
			\While {\(i < pathway\_final\_len\)}
			\While {\(j < no\_vertices\)}
			\If{checked[j] is true}
			\State {Continue;}
			\EndIf
			\State $\textit{vertex\_to\_insert} \gets \textit{-1}$
			\State $\textit{best\_cost} \gets \textit{\(\infty\)}$
			\State $\textit{vertex\_to\_insert, best\_cost, place\_to\_insert} \gets$
			\State $\textit{   Heuristic(vertex\_to\_insert, best\_cost, place\_to\_insert,  pathway, matrix, j)}$
			
			\State $\textit{place\_to\_insert} \gets \textit{-1}$
			
			\State $\textit{j} \gets \textit{j+1}$
			\EndWhile
			\State $\textit{check[vertex\_to\_insert]} \gets \textit{true}$
			\State insert\_to\_sequence($vertex\_to\_insert$ into $pathway$, on index $place\_to\_insert$)
			\State $\textit{i} \gets \textit{i+1}$
			\EndWhile
			
			\EndProcedure
		\end{algorithmic}
	\end{algorithm}
	
	\begin{algorithm}
		\caption{Heurystyka I: zachłanna}\label{zweite}
		\begin{algorithmic}[1]
			\Procedure{Greedy\_heuristic}{old\_vertex, old\_cost, old\_place, path, matrix, vertex}
			\If {$old\_cost>matrix[path[path.size-1]][vertex]$}
			\State {}
			\Return {$vertex, matrix[path[path.size-1]][vertex], path.size$}
			\EndIf
			\Return {old\_vertex, old\_cost, old\_place}
			\EndProcedure
		\end{algorithmic}
	\end{algorithm}
	
	\begin{algorithm}
		\caption{Heurystyka II: zachłanna z cyklem}\label{dritte}
		\begin{algorithmic}[1]
			\Procedure{Cycle\_heuristic}{old\_vertex, old\_cost, old\_place, path, matrix, vertex}
			\If {$old\_cost>matrix[path[path.size-1]][vertex]+matrix[path[0]][vertex]$}
			\State {}
			\Return {$vertex, matrix[path[path.size-1]][vertex]+matrix[path[0]][vertex], path.size$}
			\EndIf
			\Return {old\_vertex, old\_cost, old\_place}
			\EndProcedure
		\end{algorithmic}
	\end{algorithm}
	
	\begin{algorithm}
		\caption{Heurystyka III: 2-żal}\label{vierte}
		\begin{algorithmic}[1]
			\Procedure{2Regret\_heuristic}{old\_vertex, old\_cost, old\_place, path, matrix, vertex}
			\State $\textit{regrets} \gets \textit{empty\_sequence}$
			\State $\textit{i} \gets \textit{0}$
			
			\While {\(i < pathway\_final\_len\)}
			\State $\textit{v1} \gets \textit{path[i]}$
			\State $\textit{v2} \gets \textit{path[(i+1)\%path.size]}$
			\State append\_to\_sequence($(matrix[v1][j]+matrix[v2][j]-matrix[v1][v2], i)$ to $regrets$)
			\State $\textit{i} \gets \textit{i+1}$
			\EndWhile
			\State sort $regrets$ by first element
			\State $\textit{sum\_of\_regrets} \gets \textit{-3*regrets[0][0]}$ \Comment{Stałą wytłumaczono we wnioskach}
			\State $\textit{sum\_of\_regrets} \gets \textit{sum\_of\_regrets+regrets[1][0]+regrets[2][0]}$
			\State{}
			\If {$old\_cost > sum\_of\_regrets$}
			\State {}
			\Return {$old\_vertex, old\_cost, old\_place$}
			\EndIf
			\Return {$vertex, sum\_of\_regrets, regrets[1][0]+1$}
			\EndProcedure
		\end{algorithmic}
	\end{algorithm}
	\clearpage
	
	\begin{figure}[h!]
		\section{Rozwiązania generowane przez algorytm}
		\scalebox{.8}{
			\begin{subfigure}[ba]{1\linewidth}
				\includesvg[width=\linewidth]{heury.svg}
				\caption{Rozwiązania generowane przez algorytm}
			\end{subfigure}
		}
	\end{figure}
	\clearpage
	
	\section{Wartości rozwiązań}
	\captionof{table}{Efektywność poszczególnych heurystyk}
	\begin{tabular}{| l | l | l | l | l |}
		\rowcolor{gray!50}
		\hline
		instancja & algorytm & minimum & średnia & maksimum\\ \hline
		kroA100.tsp & zachłanny & 10298 & 12769.62 & 14877 \\ \hline
		kroA100.tsp & zachłanny z cyklem & 12308 & 14465.64 & 18139 \\ \hline
		kroA100.tsp & z 2-żalem - wersja 1 & 15042 & 16184.5 & 17235 \\ \hline
		kroA100.tsp & z 2-żalem - wersja 2 & 10450 & 11887.5 & 13625 \\ \hline
		kroB100.tsp & zachłanny & 10638 & 13521.32 & 15745 \\ \hline
		kroB100.tsp & zachłanny z cyklem & 10417 & 14148.32 & 20202 \\ \hline
		kroB100.tsp & z 2-żalem - wersja 1 & 14645 & 15672.8 & 16688 \\ \hline
		kroB100.tsp & z 2-żalem - wersja 2 & 9629 & 11346.46 & 13109 \\ \hline
		
	\end{tabular}
	
	\section{Wnioski}
	\begin{enumerate}
		\item Standardowy algorytm z 2-żalem zwracał niesatysfakcjonujące rezultaty, średnio zawsze najgorsze ze wszystkich algorytmów, na 1 z instancji najgorsze wyniki algorytmu zachłannego były lepsze niż najlepsze wyniki algorytmu z 2-żalem - wynika to zapewne z tego, że algorytm ten wybierał wierzchołek taki, że jego potencjalny wybór w przyszłości mógłby być wiele mniej opłacalny niż jest teraz; innymi słowy, wybierał wierzchołek zakładając, że on kiedyś zostanie wybrany nie uwzględniając tego, że ścieżka nie składa się ze wszystkich dostępnych wierzchołków.
		\item Ulepszony algorytm z 2-żalem odejmował od żalu naniższą możliwą wartość wzrostu odległości po dodaniu wierzchołka do ścieżki - w ten sposób maksymalny żal uwzględniał nie tylko utratę możliwości wstawienia wierzchołka w satysfakcjonujące miejsce, ale też dążył do minimalnego wzrostu długości ścieżki zamkniętej; rezultaty były średnio lepsze niż dla wszystkich pozostałych algorytmów na obu instancjach, dla instancji \textit{kroB100.tsp} najgorszy wynik tego algorytmu był lepszy niż wynik średni dla wszystkich pozostałych algorytmów.
		\item Algorytm zachłanny uzyskiwał średnio 2. najlepsze rezultaty na obu instancjach, gorsze od ulepszonej wersji algorytmu z 2-żalem, lepsze od algorytmu zachłannego tworzącego cykl. Standardowy algorytm zachłanny wybierał ścieżkę bez optymalizacji jej długości tylko raz - dla krawędzi łączącej ostatni i pierwszy wierzchołek. W przeciwieństwie do niego, algorytm zachłąnny tworzący cykl mógł wybrać taką ścieżkę wielokrotnie, ponieważ wybierał on wierzchołki, które leżą w okolicach odcinka łączącego obecnie pierwszy z obecnie ostatnim wierzchołkiem na ścieżce.
	\end{enumerate}
	Link do kodu: \color{blue} \href{https://algos.herokuapp.com}{algos.herokuapp.com}\color{black}
	
\end{document} 